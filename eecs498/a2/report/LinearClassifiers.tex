\documentclass[12pt]{article}
\usepackage{lingmacros}
\usepackage{tree-dvips}
% hyper links
\usepackage[utf8]{inputenc}
\usepackage{amsmath}
\usepackage{amsfonts}
\usepackage{bbm}
\usepackage{bm}
% Formatting quotes properly
\usepackage[english]{babel}
\usepackage[autostyle, english = american]{csquotes}
\MakeOuterQuote{"}
\usepackage{hyperref}
\usepackage{enumitem}

\usepackage[left=2cm,top=2cm,right=3cm,bottom=2.5cm,nohead,nofoot]{geometry}

\begin{document}

\noindent Benjamin Smidt

\noindent 6 September 2023
\begin{center}
\section*{EECS498 A2: Linear Classifiers}
\end{center}

\paragraph{} This documents explains the mathematics and key ideas behind my solutions 
to \emph{Assignment 2: Linear Classifiers} of Michigan's publicly available course 
\href{https://github.com/bensmidt/machine-learning/tree/main/eecs498}
{EECS 498: Deep Learning for Computer Vision}. This document is thoroughly researched 
but may not be perfect. If there's a typo or a correction needs to be made, please 
email me at benjamin.smidt@utexas.edu so I can fix it. Thank you! I hope you find this 
document helpful.

\tableofcontents{}

\newpage

\subsection*{Linear Classifiers}
Recall that our general approach requires to functions: a \textbf{score function} and a \textbf{loss function}. The score function takes the raw data (image pixels represented by a long vector) and outputs the predicted class scores. The loss function evaluates how well the scores predicted from our score function match the ground truth labels.

\subsubsection*{Definitions and Notation}
We're using the CIFAR-10 dataset with $N = 50,000$ images in our training set. Each image has $3 \times 32 \times 32 = 3072$ pixels. Each image is $32 \times 32$ in size with $3$ pixels at each location to indicate color and luminance (think RGB values). 

We'll choose to represent each image as a $3072$ dimensional vector where each index in the vector represents a single pixel value for that image. We'll denote this vector by$x_i \in \mathbb{R}^D$ where $i$ indicates the particular image in the training set of images. Since the training set contains $N$ images, $0 \le i \le N$ . We'll denote the entire training set of images by the matrix $X \in \mathbb{R}^{N \times D}$, where row $i$ of $X$ is an image $x_i$. Since this is our training set, each image $x_i$ has an associated label $y_i \in \{1, ..., K\}$ where $K$ is the number of predefined categories the image could be classified into. For instance, $K = 1$ could indicate that the image is of a dog. $K = 2$ coud indicate the image is of a car, and so on. 

\subsubsection*{Score Function}
That is, the image has $D$ number of pixels. In the CIFAR-10 dataset, an input $x_i$ would have a dimension of   

For a linear classifier it takes the form \[f(x) \]
\[ 
L_i = \sum_{j \ne y_i} max(0, s_j - s_{y_i} + \Delta) 
\]

\[ 
\frac{\partial L_i}{\partial s_j} = 
\sum_{j \ne y_i} max(0, s_j - s_{y_i} + \Delta) 
\]

\section{References}
\begin{enumerate}
    \item \href{https://cs231n.github.io/linear-classify/#softmax-classifier}{Softmax}
\end{enumerate}





\end{document}